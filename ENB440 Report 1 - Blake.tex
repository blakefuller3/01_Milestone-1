The line impedance, as calculated by CST, is seen to be 51$\Omega$, which is noticeably different to the expected value of 70$\Omega$. Initially, all geometries and material parameters were checked to ensure that this value wasn't an issue with the model. After thorough inspection no errors were found with the model.




The optimiser was used to calculate geometries needed to achieve a line impedance of 70$\Omega$. As the electrical length, height of dielectric and copper thickness all couldn't be altered, the microstrip width was chosen as the variable to optimise.\\

The optimisation software altered the original microstrip width of 0.9676mm to 0.5536mm which resulted in a line impedance of 70$\Omega$. The resulting S$_{11}$ and S$_{22}$ parameters can be seen below. There is no notable different information in the S$_{21}$ and S$_{12}$ parameters.



It can be seen that the optimisation method has matched the line impedance to the required impedance. The reflection peak has also moved closer to the operating frequency.